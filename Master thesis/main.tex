%ceci est un canevas possible pour un mémoire en français; 
%pour l'anglais (américain ou UK) c'est facile à adapter
%il y a de nombreuses variantes possibles: consulter votre promoteur ou le président de jury en cas de doute sur l'opportunité de les utiliser!
\documentclass[11pt,a4paper,oneside]{book}
\usepackage[hmargin={1.25in,1.25in},vmargin={1.25in,1.25in}]{geometry}
%%%%%%%%%%%%%%%%%%%%%%%%
\makeindex
\usepackage{textcomp}
\usepackage{fancyhdr}
\usepackage{makeidx}
\usepackage{amssymb}
\usepackage{import}
\usepackage{makeidx}
\usepackage{setspace}
\usepackage{amsmath}
\usepackage{graphicx}
\usepackage{xcolor}
\usepackage{listings}
\usepackage{footnote}
\usepackage{enumitem}
\usepackage{chapterbib,hyperref,backref}
\pagestyle{myheadings}
\fancyhf{}
\rhead[\leftmark]{thepage}
%%%%%%%%%%%%%%%%%%%%%%%%
\usepackage[utf8]{inputenc}
\usepackage[T1]{fontenc}
%\usepackage{babel}
\usepackage{url}
\usepackage{subfig}

\begin{document}
%%%%%%%%%%%%%%%%
\frontmatter

\iffalse

\begin{titlepage}
\begin{center}
\textbf{UNIVERSIT\'E LIBRE DE BRUXELLES}\\
\textbf{Faculté des Sciences}\\
\textbf{Département d'Informatique}
\vfill{}\vfill{}

%\begin{center}
{\Huge Fast dynamic predecessor queries on GPU \vspace*{.5cm}}
%\end{center}

{\Huge \par}
\begin{center}{\LARGE Youri Hubaut}\end{center}{\Huge \par}
\vfill{}\vfill{}
\begin{flushright}{\large \textbf{Promotor :} John Iacono}\hfill{}{\large Master Thesis in Computer Sciences}\end{flushright}{\large\par}
\vfill{}\vfill{}\enlargethispage{3cm}
\textbf{Academic year 2017~-~2018}
\end{center}
\end{titlepage}
\newpage
\thispagestyle{empty} 
\null

\thispagestyle{empty}
\vspace*{5cm}

\begin{quotation}
\noindent ``\emph{Celerius quam asparagi cocuntur.}''
\begin{flushright}\textbf{Gaius Suetonius Tranquillus, 87 AD}\end{flushright}
\end{quotation}

\newenvironment{vcenterpage}
{\newpage\thispagestyle{empty} 
\vspace*{\fill}}
{\vspace*{\fill}\par\pagebreak}

\chapter*{Acknowledgment}
\thispagestyle{empty} 

I would like to thank my master thesis advisor, John Iacono, for his encouragements and research ideas that led to this thesis. This work simply wouldn't have been possible without him, its critical thinking and clever remarks.\\


To Jean Cardinal, Stefan Langerman and Jan Lemeire who accepted to be members of the jury for this thesis and who aroused interest in the different aspects and themes of this work.\\


To the Computer Science departement of Université Libre de Bruxelles (U.L.B.), which provided a fertile research environment. I would like to thank its members, both students and faculty, for their valuable ideas and insights that have immensely broadened my education.\\

In closing, I would also like to thank all those who to some various degree have helped me with their interests and encouragements.

\medskip
%sans oublier les plus importants, exercice parfois délicat
\thispagestyle{empty} 
\setcounter{page}{0}
\tableofcontents
\mainmatter 

\fi

\pagenumbering{arabic}
%\import{Chapters/}{Introduction}

%\import{Chapters/GPU/}{Introduction}

%\import{Chapters/GPU/Algorithms/}{Introduction}

%\import{Chapters/Accesses/}{Accesses}

%\import{Chapters/XFastTries/}{Introduction}

%\import{Chapters/HashTable/}{Introduction}

%\import{Chapters/ParallelXFastTries/}{Introduction}

%\import{Chapters/Conclusions/}{Introduction}

\import{Chapters/Appendix/}{Introduction}

\backmatter

\printindex % il faudra utiliser l'utilitaire makeindex pour générer le fichier adéquat

\import{./}{bibliography.tex}

\end{document}

