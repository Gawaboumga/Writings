
\chapter{Chapitre premier}

En utilisant bien les références bibliographiques~\cite{ref1,ref3}.

\section{Section première}
Lorsqu'une notion importante est définie dans un gros document, il peut être utile de prévoir son introduction dans un index\index{notion importante}; cela permettra au lecteur, lors d'un usage de la notion dans le texte, de retrouver rapidement cette définition s'il l'a oubliée!
\subsection{Sous-section}
Cela permet d'obtenir un texte plus structuré
\paragraph{Titre de paragraphe}
On peut aussi avoir des subsubsections, mais il vaut en général mieux éviter de descendre trop bas avec des numérotations du type 3.2.1.5.

De même, il faut bien choisir la façon de numéroter les définitions, lemmes, propositions et théorèmes pour pouvoir bien s'y retrouver sans que ça soit trop lourd! Pour cela, on peut demander aux compteurs correspondants de fonctionner au niveau global, au niveau des sections, ... et on peut grouper les compteurs pour, par exemple, éviter d'avoir un théorème 1 qui suit une proposition 15.
\section{Section deuxième}
\section{...}Blabla~\cite{ref2}.