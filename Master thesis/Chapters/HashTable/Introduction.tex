
\chapter{Hash Table\index{Hash table}}\label{Hash table}

In this chapter, we will return to the primitive components of our data structure that are hash tables\index{Hash table}. We will come back to the different possible implementations and will try to compare them in the concurrent\index{Concurrent} framework. This will allow us to define a hash table\index{Hash table} that can be used as a base block in our X-fast trie\index{X-fast trie} to test parallelism capabilities for our data structure. We will focus mainly on insertion and search capabilities based on a single thread since this kind of operations will be the most useful to us.

\section{General context}

Hash tables\index{Hash table} play an important role in many algorithms and data structures. It is a fundamental data structure which can be used at many places. They can be seen as a bunch of slots where elements will be stored and every possible item from the universe will be mapped to one of the slots thanks to a hash function. Nevertheless, there can be collisions since the number of available slots is very much smaller than the size of the universe and thus each location can have multiple items mapped to them. Classical techniques of collision resolution must be adapted to GPUs\index{Graphics cards} due to their nature highly parallel:

\begin{itemize}
    \item Serialization: of the operations, indeed we cannot use locking mechanism on GPU\index{Graphics cards} and we must thus find solution without those and with the help of atomic\index{Atomic} operations. Those algorithms and data structures are often qualified as \textit{lock-free} if it guarantees system-wide progress or \textit{wait-free} if it also per-thread progress.
    \item Memory accesses: threads are intended to be used together to benefit from coalesced memory but the inherent nature of hash table\index{Hash table} makes it hard since the elements are randomly spread.
    \item Probing\index{Probing}: another big aspect is the warp notion, all the threads will have to wait for the slowest one, hence the worst-case number of probes required.\\
\end{itemize}

This structure is characterized by a memory-intensive task linked to the presence of numerous random accesses and, as GPUs\index{Graphics cards} present a very interesting memory bandwidth, it makes them good candidates for such structures. The goal is then to make the most of the parallelism offered, either through a data parallelism between the threads, or a work as little divergent as possible between all the elements of the same warp. These parallelism elements can also be used to simplify probing\index{Probing} by performing all accesses related to the following probes in one treatment. Finally, it is also interesting to reduce the number of atomic\index{Atomic} operations that are carried out. These induce a not insignificant cost when there are so numerous but they are also the only primitives ensuring the validity of the accesses on GPU\index{Graphics cards}.

There exist several ways to construct those data structures:

\begin{itemize}
    \item Perfect Hashing\index{Perfect hashing}: The best way to handle $O(1)$ operations and not to have troubles with collisions is to avoid collisions thanks to perfect hashing\index{Perfect hashing}. M. L. Fredman, J. Komlós and E. Szemerédi~\cite{fredman1984storing} introduced a simple construction at the price of space efficiency $O(N^{2})$. The problem consists essentially to find a hash function that will not cause collisions. There are $\binom{N}{2}$ pairs who can collide each with probability $N^{2}$, hence there is $\binom{N}{2} / N^{2} < \frac{1}{2}$ chance that there is a potential collision. Otherwise the other idea is to do a perfect hashing\index{Perfect hashing} on $O(N)$ elements and do the same thing on collisions (combining with the chaining\index{Chaining} idea), this allows to make it dynamic~\cite{dietzfelbinger1994dynamic}.
    \item Open Addressing\index{Open addressing}: A simple technique to implement, in order to insert an item, a series of probes is performed until an empty slot is found. There are three common strategies to probe: linear, quadratic and double hashing. Those probes must set the value atomically\index{Atomic} if it the key is empty, care must be taken with deletion. Problems also appear when the load factor is too important and the variance on the queries can be quite large. Resizing the table is also not trivial due to the expected lazy initialization of the new table~\cite{gao2005lock}. The expected number of probes is bounded by $1/(1 - \frac{N}{m})$ due to the hypergeometric probability where $m$ is the number of slots in total.
    \item Chaining\index{Chaining}: An alternative is to use a set of buckets and add elements in a linked-list fashion. But traversing pointers is generally not efficient. This approach can be completed by grouping the elements into larger packets~\cite{ashkiani2017dynamic} but variance can still be high if there are not enough buckets. Nevertheless, the possibilities of lock-free algorithms are numerous and they also guarantee the validity of iterators combined with resource management (garbage collector). The expected work is in $O(1 + \frac{N}{m})$ where $m$ is the number of buckets/chains. On average, for all the insertions, we get: $\frac{1}{N} \sum_{i=0}^{N-1} (1 + \frac{i}{m}) = O(1 + \frac{N}{m})$
    \item Cuckoo\index{Cuckoo hashing}: it ensures a small number of probes by limiting the number of slots an item can reside. The idea is that the elements are exchanged from one table to another each time a collision occurs. However, we may need to reconstruct the table entirely. There are some technical difficulties to implement those but they present fair results~\cite{alcantara2009real}. It is possible to show that if the load factor is less than $\frac{1}{2}$, the expected number of evictions is constant and the longest path is bounded by $O(\log N)$. Beyond, cycles may appear and there are specific solutions to solve those issues.
\end{itemize}


\section{Implementation}

We decided to implement different hash tables\index{Hash table}, in order to see, in practice, which models offered the best characteristics for our specific problem on graphic card. This also makes it possible to compare theoretical performance with that obtained in practice.
All implementations will aim to offer the same properties with the same uniqueness of elements and the same guarantees on reading/writing to be more fair; if there are simultaneous accesses (read and write) for the same key, the behaviour is undefined. We will mainly be interested in search and insertions operations per thread and not on a warp basis.

\subsection{Open addressing\index{Open addressing}}

These hash tables\index{Hash table} are very simple to implement. We start by filling them with a sentinel value representing the absence of value. When an element is inserted, based on the index obtained by the hash function and its relative offset depending on the strategy used, it is replaced atomically\index{Atomic} by the new key if the current value is still empty. However, if insertions were limited to this treatment, the results would not be as expected. Indeed, we would be able to read values that have not yet been written, which would lead to undefined behavior. We therefore chose to work with a second sentinel value that represents a key/value being inserted. The element will be accessible after the key has been changed by its final value. The tables are static, all the memory needed to hold the keys has already been pre-allocated, so you won't have to pay extra to transfer the elements from the old table to the new one in a amortized way.

\subsection{Chaining\index{Chaining}}

For the chaining\index{Chaining} policy, we have opted for a modified model where the elements are not stored in a simple linked-list but grouped by packet through a linked-list, like the idea that can be found in data structures of type deque. When several threads attempt to access this list, they attempt to atomically\index{Atomic} increment the number of items stored in the packet to determine where they should place their result. If there is not enough space left in the packet, a new node is allocated and they restart their operation to this new node. The number of buckets was determined as being twice the number of elements to insert divided by the number of elements in a linked-list node that fits on a single cache line\index{Block}.

\subsection{Cuckoo\index{Cuckoo hashing}}

Cuckoo hashing is more complicated despite a rather simple general concept, we insert an element at a certain position and if the position is already occupied, we expel it in order to replace it by our new element. The old one is then inserted into another table where the same process is repeated until the system stabilizes. The main advantage of this table is that the number of possible positions where an element can be located is limited by the number of tables used by this technique unlike open addressing\index{Open addressing} or chaining\index{Chaining}, which can lead to degenerate cases and linear complexities. Tables of the same size have been used to simplify the design but there are variants where the tables are not all of the same size.

In our case, we have opted for two different strategies to address some of the problems inherent in this hash table\index{Hash table}. Indeed, it is very likely that we fall into a cycle where we seek to insert in a loop the same subset of elements. It is then necessary to rebuild a new table by rehashing all the elements. To limit the occurrence of such problems and to avoid this significant cost, there are several solutions: the simplest is to use more than two tables, usually three or four. Each with its own hash function and we alternate cyclically between them when inserting. The other consists in creating something close to the chaining\index{Chaining}, one cell does not store only one element at a position but several of them and when there is no more place, one is expelled, generally the last one to work as a queue. Let's also mention the existence of ``stash'', a small area where we store all the keys that could not be successfully inserted~\cite{kirsch2009more} and its adaptation to GPU\index{Graphics cards}~\cite{khorasani2015stadium}.

\subsection{Remark}

All these implementations are intended to be lock-free\index{Lock-free}, unfortunately, if we insert this sentinel value representing an intermediate state synonymous with a current addition of an element, we have to wait until this operation is finished before we can continue our algorithm. This wait results in a primitive called \textit{spinlock}, an active wait on the value until a new value is set, which can greatly increase the contention and therefore the access time to the elements. Care must also be taken to pay close attention to the spinlock model used\index{SIMT}. Indeed, since threads are executed by warp, it is necessary to make sure that the processing making the situation evolve is carried out before the active wait. It is therefore necessary to play on the way the operations are scheduled in order to obtain the desired behavior and not a brutal dead lock by implementing the standard schema.



\section{Experiments}

Once we have implemented all these tables, we can start by comparing their relative performances. We have therefore set up a simple experimental protocol to test the different properties and actions that interest us. Here, we focused on three types of possible actions, inserting a new element, obtaining a value and searching for an element not present in the table. Deletion operations were not considered due to the excessive analogy of the processing carried out in relation to search operations. We then ran all the experiments in the same framework, 32 warps and 32 blocks, all threads doing the same operation on different data. Finally, we took the statistics related to the execution of 30 runs and this on 10 iterations. We decided to perform the experiments under a load factor of 50\%, it implies that we preallocate enough memory to hold twice our data, and for a number of operations between 65 thousands $(2^{15})$ up to 8 millions $(2^{23})$. We will represent the mean obtained with the standard deviation as a dot in the following.

\subsection{Insertion}

We will start by presenting the results obtained for the insertions:

\begin{figure}[!ht]
\centering
\includegraphics[width=\linewidth]{Chapters/HashTable/Results/DictionaryInsertions.png} 
\caption{Insertions in hash table\index{Hash table} 32-bit key/value}
\end{figure}

Different results can be seen on these graphs:

\begin{itemize}
    \item Cuckoo\index{Cuckoo hashing} hashing with two tables seems to present catastrophic results compared to other models. We also decided to stop its experiment after having exceeded the million of insertions seen the time necessary for the action (attention to the log plot). A very important remark must be made a priori, we present here the mean, but the median is in reality only 1.34 times larger than for the insertion with cuckoo\index{Cuckoo hashing} and 3 tables, it should lie in reality somewhere between 3 and 4 tables. This also implies that one can fall in strongly degenerated cases where the active waiting leads to much contention, which makes explode the execution time.
    \item The very essence of cuckoo\index{Cuckoo hashing} operation leads to a much stronger sequentiality\index{Sequential} of actions, and as one may be led to transfer many table elements to table, this induces very strong adverse cases. A more adapted tuning of the parameters as well on the level of the size of the tables as in the functions of hashage employed can partially mitigate these problems.
    \item Chaining\index{Chaining} proposes surprisingly low performances which can be explained by the mode of access to the data and the indirection necessary to access the data, one has a low efficiency in the use of the cache\index{Cache}.
    \item We can also note that the cuckoo\index{Cuckoo hashing} with more than 2 tables and buckets of size 4 leads to a noticeable acceleration of performance. So, once a cached line\index{Block} has been loaded, its access becomes essentially free.
    \item Finally, those who use notions related to open addressing\index{Open addressing} fare best. They even fit in a pocket handkerchief; no noticeable difference, mainly due to the relatively low load factor, 50\%.
\end{itemize}

\subsection{Search}

We can then move on to the two search operations, one successful and the other looking for an element that does not exist:

\begin{figure}[!ht]
\centering
  \includegraphics[width=\linewidth]{Chapters/HashTable/Results/DictionaryGet.png}
  \captionof{figure}{Successful search}
  \label{fig:successful_get}
 \end{figure}
 
\begin{figure}[!ht]
  \centering
  \includegraphics[width=\linewidth]{Chapters/HashTable/Results/DictionaryGetUnsuccessful.png}
  \captionof{figure}{Unsuccessful search}
  \label{fig:unsuccessful_get}
\end{figure}

Again, many remarks to be made:
\begin{itemize}
    \item Chaining\index{Chaining} and cuckoo\index{Cuckoo hashing} with bucket present the worst results of all these techniques. The performance is quite logical since these techniques are based on a similar scheme where all elements associated with a value must be observed before moving to the next node/table. A lot of processing is then done and the divergence in the same warp is all the greater as the number of possible branching may exist.
    \item When there are few elements, cuckoo hashing\index{Cuckoo hashing} seems to offer a real alternative to open addressing\index{Open addressing} with a lower standard deviation and a slightly lower but still significant average (by unpaired t-test). But on a larger number the interest seems tenuous. Indeed, the locality of the data and their relative efficiency play a crucial role in access times. Accessing the entire cache line\index{Block} even if loaded randomly and concurrently by each thread remains comparable to lower efficiency but with sequential\index{Sequential} accesses.
    \item A notable and curious difference exists between the two cases of searches. If the element exists, cuckoo\index{Cuckoo hashing} with 2 tables shows the best access times. On the contrary, if the element is absent, 3 tables seem slightly more effective. This phenomenon is somewhat bizarre and difficult to explain. It seems logical that searching in fewer tables takes less time since you have to load less data in the end but the other case is a real mystery.
    \item Finally, a linear probing\index{Probing} seems to be the best in our case (50\% of load factors), we then find quite naturally quadratic and double hashing since quadratic offers at the beginning, a greater probability to remain in the same cache line\index{Block}, whereas with double hashing, it becomes practically nil. It is funny to note that cuckoo hashing\index{Cuckoo hashing} and double hashing seem to offer the same performance, which would indicate that an equivalent number of data are read in both cases, i.e. 2.
\end{itemize}

\subsection{Conclusions}

Cuckoo hashing\index{Cuckoo hashing} with just two tables is not a viable alternative in the context of graphics cards\index{Graphics cards}. Unless the static insertion approach is chosen as originally presented~\cite{alcantara2009real}. Working with 3 tables (at least) or with buckets seems a much better alternative but buckets have the disadvantage of having relatively slow access times to the elements and, in the end, one would prefer to use 3 tables if the cuckoo\index{Cuckoo hashing} approach is really necessary. This has the advantage of having a bounded time on search operations compared to the other two families and the standard deviation is slightly lower, which is better suited to real time conditions. It is also the technique that can afford a high load factor without paying too much on the resulting performances.

Chaining\index{Chaining} does not seem to offer any interest by itself, if we obviously omit the possibilities of resizing the data structure and its flexibility in memory. It is much easier to offer strong guarantees on the validity of operators in this context.

Open addressing\index{Open addressing} seems to be the big winner of our test, both in terms of insertions and searches. However, the removal policy is a little more complex and it may be necessary to clean this table regularly, which induces a cost in fine. Hopscotch hashing~\cite{herlihy2008hopscotch} presents a very interesting variant to these techniques since it consists in limiting the region in which an element can be located by exchanging the position of two elements as long as they remain sufficiently close to their original position. This would introduce a cost on insertions but would help reduce the worst case in search operations. Moreover, the linear variant of open addressing\index{Open addressing} seems to win hands down in each of the tested conditions, we will opt for it when implementing our X-fast trie\index{X-fast trie}.

Finally, the results are more or less the same using 64-bit keys, but there is a greater decrease in performance for linear open addressing\index{Open addressing} than in any other case due to the greater number of memory requests made. But the latter remains a winner in all cases. Finally, in order to give perspectives on the possible performances obtained, our implementation allows the insertion of 325 million elements per second and to recover the value associated with 572 million keys per second. See also, Heer et al.~\cite{heerdata} which regroup all the results obtained for hash tables\index{Hash table} on GPU\index{Graphics cards}.
