
\chapter{Parallel X-fast tries}\label{PARALLELXFASTTRIE}

We have previously seen in detail the X-fast tries\index{X-fast trie}, their general concept and implementation strategy, in a sequential framework$^{[\ref{XFastTrie}]}$. We can try to adapt it better to the paradigm of graphic cards\index{Graphics cards} and thus take advantage of all the computing power offered by them. Indeed, our implementation was for the moment purely sequential\index{Sequential} in order to see if it presented only a theoretical or real interest. Following the results, we looked at the question of making this data structure concurrent\index{Concurrent} and since they are based on hash tables\index{Hash table}, it was interesting to come back to these different notions. In this chapter, we will attempt to combine these two aspects (the X-fast tries\index{X-fast trie} and concurrent\index{Concurrent} hash tables\index{Hash table}) into one to see what could be achieved for more exploitable purposes. This will also allow us to determine whether this structure is still relevant in a highly parallel context.


\section{General ideas}

Let's take two minutes to ponder on what is absolutely necessary to make our data structure concurrent\index{Concurrent}. When we insert an element, we must start by looking for the lowest node in the tree where the difference is made, where the common prefix starts to diverge between the element we want to add and all those already inserted. First, update all parent nodes as long as necessary in order to always keep the minimum and maximum value of the entire subtree, a way to access the predecessor or successor in a constant time. Second, insert all nodes that will form the path to the leaf with the increasing prefix. Finally, it remains to insert the element at leaf level and update its direct predecessor and successor.

In the remainder of this chapter, we will consider essentially bulk operations, we will concentrate first on a set of the same operations carried out simultaneously and not a mixture of these. That is to say, we will later be interested in reading, inserting and deleting operations that will take place simultaneously, interleaved.

As a preamble, we will point out that reading / writing in the different levels of the tree cannot be done simultaneously, a certain latitude is allowed due to the different positions at which an element can be located in a hash table\index{Hash table}, hence a variable time to retrieve it. This is probably the most annoying aspect of proving that the invariants are well respected.

Look for the lowest level in the tree shouldn't be a problem. Indeed, this step is limited to the membership property and it is very likely that the previous elements are fully inserted in the tree. The only embarrassing case is when levels have not yet been inserted or are currently being inserted and only a subset of them can be determined to exist. Only, this one has a reduced impact, since we only look where the split occurs. But we must keep in mind that we can insert several elements belonging to the same subtree at the same time, or more specifically, several times the same element.

Updating parents should not be too difficult, either we practice a consistent reading policy on hash tables\index{Hash table}, that is, we can only read inserted elements, in this case, we are certain that parents exist and it is enough to update the minimum and maximum values of their subtree atomically\index{Atomic}. Either, when we read information in the hash table\index{Hash table}, we pass directly those being inserted and we do not wait for them to obtain their final value. This case can be more annoying since we can have a subset of the parent elements, we must then complete those missing or wait until they have a final value to update them.

Completing the subtree to mark the path to the leaves is not a problem as long as a special policy is put in place in the hash tables\index{Hash table}. The most attentive will have noticed the previous paragraph had the same defect and was subtly erroneous. We have completed the hash table we use to insert a new feature that is found in various implementations and is usually called \textit{upsert} (compound word for ``insert or update''). This consists simply in carrying out the same treatment as the insertion with the exception if the element is already present; in this case, a function is applied on the basis of the old and the new value and the result is written. The insertion is thus a particular case of this operation which consists only in writing the new value and discarding the old one.

The real difficulty arises when you want to insert an element at leaf level. Indeed, the operation is not limited to a simple insertion since we must update the predecessor and the successor in order to keep a coherence in the equivalent of this double linked and ordered list. This operation is far from being trivial and requires careful attention to the invariants that one wishes to keep at all times. A simple solution is to use the non-blocking\index{Lock-free} technique of the philosophers' dinner, \textit{compare and swap}\index{Compare and swap}\index{Atomic} as long as we have not blocked our two neighbors and update this information~\cite{tanenbaum2009modern}. However, this is not enough because a precedent or successor may have been inserted in the meantime, so care must be taken to keep the smallest interval and block only those. In our case, we considered a slightly different solution where we tried to define ourselves as the predecessor and successor thanks to the ``compare and swap''\index{Compare and swap} instruction.



\section{Log structured merge tree (LSM)\index{LSM}}

After having finished our implementation, we wanted to compare it with another one in order to have a comparison on which to base ourselves and to note or not the interest of such a structure. We looked for a comparable data structure that offers the same features as a dictionary\index{Dictionnary} and that exists on the GPU\index{Graphics cards}. The choice was not long to decide since there is only one other offering these operations. There are works on Bounding Volume Hierarchy or R-trees but beyond the hash tables\index{Hash table}, it is somewhat lean cow. There are very few data structures specifically designed for such devices.

The Log Structured Merge tree (which we will name later LSM\index{LSM}) is a dynamic dictionary\index{Dictionnary} data structure for the GPU\index{Graphics cards}. It was developed at the end of 2017 by S. Ashkiani, S. Li, M. Farach-Colton, N. and J. D. Owens as part of a doctoral thesis of the main author~\cite{ashkiani2017parallel}. It aims to address three issues. First, it is very difficult to maintain a data structure in a dynamic context, some of the literature is devoted to creating static structure~\cite{alcantara2011building, popov2007stackless}, so a total reconstruction is re-done each time; here, updating the structure remains competitive. Second, it aims to propose operations related to dictionaries\index{Dictionnary}, namely the canonical ones: insertion/deletion/look up as well as range and count which can be seen as predecessor queries if we omit one bound. Third, exploit the capabilities of the graphics card\index{Graphics cards} while ensuring the correctness of the structure. The invariant proposed by this data structure is very strong.

This data structure is the result of the fusion of two concepts: the Log-structured Merge-tree (LSM)\index{LSM}~\cite{o1996log} and the Cache Oblivious Lookahead Array (COLA)\index{Cache oblivious}~\cite{bender2007cache}. The basic idea of a LSM\index{LSM} is to contain a set of dictionaries\index{Dictionnary} of increasing size. The elements are inserted in the first dictionary\index{Dictionnary} and when it is complete, its content is merged with the next one in the list and so on. This has two consequences, to search for an element, you have to look in each dictionary\index{Dictionnary}, which induces a cost of $O(\log N \log_{B} N)$, but to insert elements, you benefit more from the locality of the data since you merge two adjoining levels. In practice, the insertions are faster than for a B-Tree but the queries are slower. COLA, on the other hand, consists simply in having an array sorted instead of dictionaries\index{Dictionnary}.

The best way to exploit the parallelism offered by graphics cards\index{Graphics cards} for such a data structure is to work by batch. We therefore set a parameter $b$ which corresponds to the number of elements we want to insert at each step and which is equal to the size of the first buffer in this structure. Both update operations are performed by batch, insertions and deletions. Queries do not depend on this parameter and can be performed by a single thread.

\begin{figure}[!htb]
    \centering
    \includegraphics[width=0.75\linewidth]{Chapters/ParallelXFastTries/LSM.png} 
    \caption{Batch insertion with already 5 inserted batches - image extracted from Ashkiani et al.~\cite{ashkiani2017gpu}, arXiv:1707.05354}
\end{figure}

Let's first clarify that the different dictionaries\index{Dictionnary} used internally by this data structure have sizes that are powers of 2 and multiples of $b$ (i.e. $b2^{i}$). To insert or delete elements, we first sort the data related to the batch. Then, we always apply the same procedure, we check if the $i$th level is complete or empty (they cannot be partially filled), if it is empty, we insert all our elements. Otherwise, the elements to be inserted are merged with the elements of the $i$th level, the two containers become one and the previously filled level is emptied. We then try to insert the result in the next container, and so on until we find an empty dictionary\index{Dictionnary}. It should be noted that the filled levels correspond to the bit set to 1 in the binary representation of the number of batches inserted $r$, and 0 for the empty ones. There are thus $n = br$ elements in the structure. The total amount of work to insert $n$ elements is thus $O(rb \log r)$ since the worst case is complete cascading $O(rb)$ and this occurs with $O(i) \leq O(\log r)$. Observe that duplicates may exist naturally and that deletions correspond to sentinel values sharing the same key. In their implementation, they propose to set the most significant bit to 1 to indicate a deletion, which will serve in the merge procedure.

Queries can be simply implemented through classic lower/upper bound queries on each filled level and leading to natural $O(\log^{2} n)$. Note that merge procedures are somewhat difficult to implement in a parallel context, especially when duplicates exist~\cite{green2012gpu}.


\section{Implementation}

We decided to implement different hash tables\index{Hash table}, in order to see, in practice, which models offered the best characteristics for our specific problem on graphic card. This also makes it possible to compare theoretical performance with that obtained in practice.
All implementations will aim to offer the same properties with the same uniqueness of elements and the same guarantees on reading/writing to be more fair; if there are simultaneous accesses (read and write) for the same key, the behaviour is undefined. We will mainly be interested in search and insertions operations per thread and not on a warp basis.

\subsection{Open addressing\index{Open addressing}}

These hash tables\index{Hash table} are very simple to implement. We start by filling them with a sentinel value representing the absence of value. When an element is inserted, based on the index obtained by the hash function and its relative offset depending on the strategy used, it is replaced atomically\index{Atomic} by the new key if the current value is still empty. However, if insertions were limited to this treatment, the results would not be as expected. Indeed, we would be able to read values that have not yet been written, which would lead to undefined behavior. We therefore chose to work with a second sentinel value that represents a key/value being inserted. The element will be accessible after the key has been changed by its final value. The tables are static, all the memory needed to hold the keys has already been pre-allocated, so you won't have to pay extra to transfer the elements from the old table to the new one in a amortized way.

\subsection{Chaining\index{Chaining}}

For the chaining\index{Chaining} policy, we have opted for a modified model where the elements are not stored in a simple linked-list but grouped by packet through a linked-list, like the idea that can be found in data structures of type deque. When several threads attempt to access this list, they attempt to atomically\index{Atomic} increment the number of items stored in the packet to determine where they should place their result. If there is not enough space left in the packet, a new node is allocated and they restart their operation to this new node. The number of buckets was determined as being twice the number of elements to insert divided by the number of elements in a linked-list node that fits on a single cache line\index{Block}.

\subsection{Cuckoo\index{Cuckoo hashing}}

Cuckoo hashing is more complicated despite a rather simple general concept, we insert an element at a certain position and if the position is already occupied, we expel it in order to replace it by our new element. The old one is then inserted into another table where the same process is repeated until the system stabilizes. The main advantage of this table is that the number of possible positions where an element can be located is limited by the number of tables used by this technique unlike open addressing\index{Open addressing} or chaining\index{Chaining}, which can lead to degenerate cases and linear complexities. Tables of the same size have been used to simplify the design but there are variants where the tables are not all of the same size.

In our case, we have opted for two different strategies to address some of the problems inherent in this hash table\index{Hash table}. Indeed, it is very likely that we fall into a cycle where we seek to insert in a loop the same subset of elements. It is then necessary to rebuild a new table by rehashing all the elements. To limit the occurrence of such problems and to avoid this significant cost, there are several solutions: the simplest is to use more than two tables, usually three or four. Each with its own hash function and we alternate cyclically between them when inserting. The other consists in creating something close to the chaining\index{Chaining}, one cell does not store only one element at a position but several of them and when there is no more place, one is expelled, generally the last one to work as a queue. Let's also mention the existence of ``stash'', a small area where we store all the keys that could not be successfully inserted~\cite{kirsch2009more} and its adaptation to GPU\index{Graphics cards}~\cite{khorasani2015stadium}.

\subsection{Remark}

All these implementations are intended to be lock-free\index{Lock-free}, unfortunately, if we insert this sentinel value representing an intermediate state synonymous with a current addition of an element, we have to wait until this operation is finished before we can continue our algorithm. This wait results in a primitive called \textit{spinlock}, an active wait on the value until a new value is set, which can greatly increase the contention and therefore the access time to the elements. Care must also be taken to pay close attention to the spinlock model used\index{SIMT}. Indeed, since threads are executed by warp, it is necessary to make sure that the processing making the situation evolve is carried out before the active wait. It is therefore necessary to play on the way the operations are scheduled in order to obtain the desired behavior and not a brutal dead lock by implementing the standard schema.



\section{Experiments}

Once we have implemented all these tables, we can start by comparing their relative performances. We have therefore set up a simple experimental protocol to test the different properties and actions that interest us. Here, we focused on three types of possible actions, inserting a new element, obtaining a value and searching for an element not present in the table. Deletion operations were not considered due to the excessive analogy of the processing carried out in relation to search operations. We then ran all the experiments in the same framework, 32 warps and 32 blocks, all threads doing the same operation on different data. Finally, we took the statistics related to the execution of 30 runs and this on 10 iterations. We decided to perform the experiments under a load factor of 50\%, it implies that we preallocate enough memory to hold twice our data, and for a number of operations between 65 thousands $(2^{15})$ up to 8 millions $(2^{23})$. We will represent the mean obtained with the standard deviation as a dot in the following.

\subsection{Insertion}

We will start by presenting the results obtained for the insertions:

\begin{figure}[!ht]
\centering
\includegraphics[width=\linewidth]{Chapters/HashTable/Results/DictionaryInsertions.png} 
\caption{Insertions in hash table\index{Hash table} 32-bit key/value}
\end{figure}

Different results can be seen on these graphs:

\begin{itemize}
    \item Cuckoo\index{Cuckoo hashing} hashing with two tables seems to present catastrophic results compared to other models. We also decided to stop its experiment after having exceeded the million of insertions seen the time necessary for the action (attention to the log plot). A very important remark must be made a priori, we present here the mean, but the median is in reality only 1.34 times larger than for the insertion with cuckoo\index{Cuckoo hashing} and 3 tables, it should lie in reality somewhere between 3 and 4 tables. This also implies that one can fall in strongly degenerated cases where the active waiting leads to much contention, which makes explode the execution time.
    \item The very essence of cuckoo\index{Cuckoo hashing} operation leads to a much stronger sequentiality\index{Sequential} of actions, and as one may be led to transfer many table elements to table, this induces very strong adverse cases. A more adapted tuning of the parameters as well on the level of the size of the tables as in the functions of hashage employed can partially mitigate these problems.
    \item Chaining\index{Chaining} proposes surprisingly low performances which can be explained by the mode of access to the data and the indirection necessary to access the data, one has a low efficiency in the use of the cache\index{Cache}.
    \item We can also note that the cuckoo\index{Cuckoo hashing} with more than 2 tables and buckets of size 4 leads to a noticeable acceleration of performance. So, once a cached line\index{Block} has been loaded, its access becomes essentially free.
    \item Finally, those who use notions related to open addressing\index{Open addressing} fare best. They even fit in a pocket handkerchief; no noticeable difference, mainly due to the relatively low load factor, 50\%.
\end{itemize}

\subsection{Search}

We can then move on to the two search operations, one successful and the other looking for an element that does not exist:

\begin{figure}[!ht]
\centering
  \includegraphics[width=\linewidth]{Chapters/HashTable/Results/DictionaryGet.png}
  \captionof{figure}{Successful search}
  \label{fig:successful_get}
 \end{figure}
 
\begin{figure}[!ht]
  \centering
  \includegraphics[width=\linewidth]{Chapters/HashTable/Results/DictionaryGetUnsuccessful.png}
  \captionof{figure}{Unsuccessful search}
  \label{fig:unsuccessful_get}
\end{figure}

Again, many remarks to be made:
\begin{itemize}
    \item Chaining\index{Chaining} and cuckoo\index{Cuckoo hashing} with bucket present the worst results of all these techniques. The performance is quite logical since these techniques are based on a similar scheme where all elements associated with a value must be observed before moving to the next node/table. A lot of processing is then done and the divergence in the same warp is all the greater as the number of possible branching may exist.
    \item When there are few elements, cuckoo hashing\index{Cuckoo hashing} seems to offer a real alternative to open addressing\index{Open addressing} with a lower standard deviation and a slightly lower but still significant average (by unpaired t-test). But on a larger number the interest seems tenuous. Indeed, the locality of the data and their relative efficiency play a crucial role in access times. Accessing the entire cache line\index{Block} even if loaded randomly and concurrently by each thread remains comparable to lower efficiency but with sequential\index{Sequential} accesses.
    \item A notable and curious difference exists between the two cases of searches. If the element exists, cuckoo\index{Cuckoo hashing} with 2 tables shows the best access times. On the contrary, if the element is absent, 3 tables seem slightly more effective. This phenomenon is somewhat bizarre and difficult to explain. It seems logical that searching in fewer tables takes less time since you have to load less data in the end but the other case is a real mystery.
    \item Finally, a linear probing\index{Probing} seems to be the best in our case (50\% of load factors), we then find quite naturally quadratic and double hashing since quadratic offers at the beginning, a greater probability to remain in the same cache line\index{Block}, whereas with double hashing, it becomes practically nil. It is funny to note that cuckoo hashing\index{Cuckoo hashing} and double hashing seem to offer the same performance, which would indicate that an equivalent number of data are read in both cases, i.e. 2.
\end{itemize}

\subsection{Conclusions}

Cuckoo hashing\index{Cuckoo hashing} with just two tables is not a viable alternative in the context of graphics cards\index{Graphics cards}. Unless the static insertion approach is chosen as originally presented~\cite{alcantara2009real}. Working with 3 tables (at least) or with buckets seems a much better alternative but buckets have the disadvantage of having relatively slow access times to the elements and, in the end, one would prefer to use 3 tables if the cuckoo\index{Cuckoo hashing} approach is really necessary. This has the advantage of having a bounded time on search operations compared to the other two families and the standard deviation is slightly lower, which is better suited to real time conditions. It is also the technique that can afford a high load factor without paying too much on the resulting performances.

Chaining\index{Chaining} does not seem to offer any interest by itself, if we obviously omit the possibilities of resizing the data structure and its flexibility in memory. It is much easier to offer strong guarantees on the validity of operators in this context.

Open addressing\index{Open addressing} seems to be the big winner of our test, both in terms of insertions and searches. However, the removal policy is a little more complex and it may be necessary to clean this table regularly, which induces a cost in fine. Hopscotch hashing~\cite{herlihy2008hopscotch} presents a very interesting variant to these techniques since it consists in limiting the region in which an element can be located by exchanging the position of two elements as long as they remain sufficiently close to their original position. This would introduce a cost on insertions but would help reduce the worst case in search operations. Moreover, the linear variant of open addressing\index{Open addressing} seems to win hands down in each of the tested conditions, we will opt for it when implementing our X-fast trie\index{X-fast trie}.

Finally, the results are more or less the same using 64-bit keys, but there is a greater decrease in performance for linear open addressing\index{Open addressing} than in any other case due to the greater number of memory requests made. But the latter remains a winner in all cases. Finally, in order to give perspectives on the possible performances obtained, our implementation allows the insertion of 325 million elements per second and to recover the value associated with 572 million keys per second. See also, Heer et al.~\cite{heerdata} which regroup all the results obtained for hash tables\index{Hash table} on GPU\index{Graphics cards}.



\section{Speed-up}

After collecting all these data and results, we asked ourselves what was the scalability of such a data structure. And mainly, what was the influence of the number of blocks and warps on the performances. Indeed, with material advances, we can expect an increase in these resources both in quantity and in speed (cost related to latency, scheduling,...). Will it remain interesting in a more distant future.

So we collected data on the insertion of 131 thousand items ($2^{17}$) for many configurations. We varied the number of blocks and warps from 1 to 32 and from 1 to 16 respectively by doubling each time their number. We present the mean time associated with its standard deviation (in parenthesis). The intermediate lines represent the ratio between the time taken for that specific configuration and that with only one block and one warp (and therefore non-concurrent).

\begin{table}[]
\centering
\caption{Time to insert 32-bit keys in function of the number of warps and blocks}
\label{my-label}
\resizebox{\textwidth}{!}{
\begin{tabular}{ccccccc}
         & \# Warps 1           & 2                    & 4                    & 8                   & 16                 \\
\#Blocks &                      &                      &                      &                     &                    \\
1        & 611080.18 (2151.349) & 306842.126 (970.405) & 154465.447 (203.045) & 80001.789 (163.276) & 42918.908 (81.063) \\
         & 100.0                & 50.21                & 25.28                & 13.09               & 7.02               \\
2        & 316926.742 (789.772) & 159116.154 (579.524) & 80069.378 (101.706)  & 41476.388 (98.77)   & 22452.94 (44.331)  \\
         & 51.86                & 26.04                & 13.1                 & 6.79                & 3.67               \\
4        & 160547.17 (356.854)  & 80964.003 (269.402)  & 40692.377 (88.208)   & 21198.831 (38.379)  & 11562.582 (31.429) \\
         & 26.27                & 13.25                & 6.66                 & 3.47                & 1.89               \\
8        & 81903.641 (182.339)  & 41682.092 (131.041)  & 21026.405 (63.426)   & 11649.702 (36.151)  & 7061.199 (49.369)  \\
         & 13.4                 & 6.82                 & 3.44                 & 1.91                & 1.16               \\
16       & 41745.251 (124.893)  & 21266.58 (90.145)    & 11281.267 (51.34)    & 6674.541 (30.823)   & 7487.541 (43.459)  \\
         & 6.83                 & 3.48                 & 1.85                 & 1.09                & 1.23               \\
32       & 21303.918 (62.376)   & 11469.453 (34.3)     & 6597.327 (17.209)    & 7461.54 (32.463)    & 5709.916 (33.266)  \\
         & 3.49                 & 1.88                 & 1.08                 & 1.22                & 0.93                 
\end{tabular}
}
\end{table}

The results are self-explanatory:
\begin{itemize}
    \item Generally speaking, we see the times reduced with the increase in parallelism capacities.
    \item The times are roughly equivalent on the diagonals, which is relatively logical. And implies that both the increase in the number of blocks and warps contribute to the overall performance improvement.
    \item The speed-up is far from being linear, we get more than 20\% of variation from the theoretical increase for configurations such as $\#warps \times \#blocks \geq 128$. And we're almost at a factor $2$ on the last configuration proposed where we reach nearly the maximum possible speed. This may be explained by the fact that we are beginning to reach the maximum possible bandwidth and that the concurrency problems on the double-linked list at the leaf level are not totally without cost.
\end{itemize}




\section{Interleaved operations}

Finally, we would like to point out that we have only considered bulk operations, where only one type of operation is carried out. This case is obviously very far from reality where a mixture is by no means exceptional. We would therefore like to come back to this point.

We clearly expect that interleaving operations and proving that implementations are correct will not be an easy task. Some mixtures don't seem to pose much of a problem at first glance. Indeed, all those who look for an element in the structure correspond to the classic case of hash tables, hence, we have the standard guarantees. Inserting elements and searching for a predecessor/successor doesn't seem too difficult either if you allow yourself a certain flexibility on the results you want to obtain.

The real concerns come with the removal of elements. Indeed, both at the level of the leaves and the maintenance of the sorted list and at the levels of the intermediate leaves and the maximum/minimum values of the subtree, problems can arise. Perhaps invalidating elements and rebuilding the structure from time to time would be a better solution.


\section{Conclusions}

As we have seen, the results we obtained in the sequential\index{Sequential} framework are surprising close from those obtained in the concurrent\index{Concurrent} framework. Of course, we expected a certain loss of performance related to the very large number of memory accesses needed to respond to requests, unlike the very limited subset proposed by the LSM\index{LSM}, but clearly not in such measures.

The theoretical performance offered by X-fast tries\index{X-fast trie} was really important to check and adapt to the context of the graphics cards\index{Graphics cards}. This data structure proposes a real alternative to LSM. We want to emphasize that we are far from the number of insertions being announced in their article since we used smaller batch sizes, but our data structure has the advantage of offering insertions by element.

One of the big black spots of this data structure is obviously the need to abuse atomic\index{Atomic} operations, and these are inherently terribly slow. We count no less than 7 million atomic transactions to insert a hundred thousand elements, however the system can provide more than 7GB/s for these operations. Nonetheless, the performances remain interesting. The other one is the amount of memory needed to hold the structure in memory which is incredibly huge. We stopped our experiments for more than 250 thousands ($2^{18}$) of insertions since we reach the limits of our available memory.

To resume, we achieve equivalent performance for insertions (10Mop/s) and thread-based searches (300Mop/s). We are faster on searches that use an entire warp (110Mop/s vs 30Mop/s) and we are slower on average on predecessor queries (10Mop/s vs 30Mop/s). Our data structure is of interest if the number of predecessor/successor requests remains lower than the search operations.

Nevertheless, we obtained these results in an ideal case, where all the space had been pre-allocated. One expects to become notoriously slower on insertions in practice. The load factor used in hash tables is relatively low 50\%, one can also lose efficiency there.

Besides, it has the advantage of being able to improve itself thanks to scientific or material advances. It also explores a path in research where parallelism can be used at lower cost to find elements effectively. We can hope to see other more advanced structures employing similar techniques, like warp election.

To conclude, we have tried to group in a single table the main differences proposed for the only two existing dynamic dictionaries\index{Dictionnary} on GPUs\index{Graphics cards}, at this time, in order to have a more theoretical and high level aspect of what is proposed in practice.

\newpage

\begin{table}[!htb]
\centering
\caption{Resume comparison of LSM\index{LSM} and X-Fast tries}
\label{my-label}
\begin{tabular}{ccc}
                     & LSM                  & X-Fast tries          \\
Insertion            & Bulk and Block       & Element and Warp       \\
$w = |\text{warp size}|$ & $O(\log N)$          & $O(\frac{\log u}{w})$ \\
\multicolumn{1}{l}{} & \multicolumn{1}{l}{} & \multicolumn{1}{l}{}  \\
Search               & Thread               & Thread                \\
                     & $O(\log^{2} N)$      & $O(1)$                \\
\multicolumn{1}{l}{} & \multicolumn{1}{l}{} & \multicolumn{1}{l}{}  \\
Predecessor          & Thread               & Warp                  \\
                     & $O(\log^{2} N)$      & $O(\frac{\log u}{w})$ \\
\multicolumn{1}{l}{} & \multicolumn{1}{l}{} & \multicolumn{1}{l}{}  \\
Range queries        & Yes                  & No\footnotemark       \\
$L$ output           & $O(\log^{2} N + L)$  & $O(L)$                \\
\multicolumn{1}{l}{} & \multicolumn{1}{l}{} & \multicolumn{1}{l}{}  \\
Concurrency          & Difficult            & Medium                \\
Memory               & Proportional (3N)    & Huge (+100N)$^{2}$    \\
Variance             & High                 & Small                 \\
Requirements         & Comparable keys      & Only integers         \\
Implementation       & Easy                 & Hard          
\end{tabular}
\end{table}
\footnotetext{Lazy iteration on the elements.\\$~\quad~^{2}$Proportional to $\Theta(3 N \log u)$.}

% TALK ABOUT LINEARIZABILITY


%http://people.cs.georgetown.edu/~jfineman/papers/cobtree.pdf
%Correctness
%We first argue that the data structure is correct, even under concurrent
%operations. The most common way of showing that an algorithm
%implements a linearizable object is to show that in every
%2The hyperceiling of x, denoted ddxee, is defined to be 2
%dlogxe
%, i.e.,
%the smallest power of 2 greater than x.
%execution there exists a total ordering of the operations with the
%following properties: (1) the ordering is consistent with the desired
%insert/search semantics, and (2) if one operation completes before
%another begins, then the first operation precedes the second in the
%ordering. Linearizability follows due to a straightforward extension
%of Lemmas 13.10 and 13.16 in [22].