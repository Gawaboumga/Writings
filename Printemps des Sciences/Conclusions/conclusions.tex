\chapter{Conclusions et perspectives}

\section{Conclusions et perspectives}

Vous l'aurez remarqué, les systèmes de recommandations représentent un vaste pan de l'informatique. Nous avons abordé dans ce travail que deux aspects: celui des recommandations dites basées sur le milieu (\textit{Knowledge-based}) et l'analyse sémantique latente. Nous aurions tant aimé aborder d'autres sujets mais nous sommes limités tant par l'espace que par le temps.

Dans ce projet, nous avons vu qu'il était possible d'implémenter un système de recommandations, avec des concepts certes complexes ou plutôt tardifs dans l'histoire des Sciences, en peu de lignes de codes. Cependant, il ne s'agit que d'une simple ébauche, d'un premier aperçu de toute la complexité de cette problématique. Une première approximation est relativement triviale à implémenter mais l'amélioration des résultats ne l'est certainement pas. Nous comprenons mieux que des entreprises telles que Google ou Amazon prennent des années à se développer pour arriver à de tels résultats et devenir des géants tous puissants de l'économie de marché.

Nous espérons que la lecture de ce sujet vous aura passionné autant que nous. Nous vous invitons à vous renseigner d'avantage dans ces théories qui sont porteuses de beaucoup de découvertes et qui ont attrait à différents aspects de la Science en général. La recherche dans ce domaine vous tend les bras et ne demande que votre aide. Des concours sont organisés de manière plus casuelle afin que les gens puissent effleurer du bout du doigt ce champ de l'informatique et de véritables challenges sont créés par des entreprises afin d'améliorer leur propre système avec des lots intéressants à la clef.