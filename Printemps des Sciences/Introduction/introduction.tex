\chapter{Introduction}

\section{Plan}

Dans ce rapport, nous présentons le travail que nous avons effectué pour le \textit{Printemps des Sciences}. Nous commencerons par établir la problématique que nous étudierons, nous donnerons ensuite un bref aperçu des diverses technologies qui permettent de répondre à celle-ci. Nous continuerons sur la technique que nous avons décidé d'employer et de développer. Enfin, nous terminerons par une analyse de notre application au travers de tests que nous tenterons de critiquer et d'interpréter ainsi que par proposer d'autres procédés afin d'aller plus loin.

\section{Problématique}

Le thème de ce \textit{Printemps des Sciences} est l'alimentation, nous avons donc choisi un sujet en accord, celui du gaspillage alimentaire. Cette problématique a pris de plus en plus d'ampleur ces dernières années et il est important de lutter contre ses répercussions. En effet, ce sujet de préoccupation a beaucoup de conséquences néfastes sur la société en général et son mode de consommation. Alors que des personnes continuent de mourir de faim dans les pays en voie de développement, nous sommes parfois obligés de jeter de la nourriture à la poubelle ! L'industrie agroalimentaire ne sait pas toujours comment récupérer ses déchets afin de les retransformer à des fins utiles comme de l'engrais ou de l'énergie. L'économie et notre portefeuille sont également touchés par ces biens qui ont perdu toute leur valeur et ne contribue donc plus au cycle dans la société.

\section{Easy Fridge}

Nous avons donc développé une application pour téléphone portable afin de lutter contre le gaspillage alimentaire domestique. Nous essayons d'endiguer les principales causes de ce phénomène que sont:
\begin{itemize}
    \item La mauvaise gestion du réfrigérateur, par le biais d'une classification erronée des produits selon leur date de péremption.
    \item Le manque d'idées quant à l'utilisation des denrées alimentaires.
    \item Les achats inutiles que nous effectuons parce que nous avons succombé à des promotions.
\end{itemize}

Pour cela, nous proposons d'avoir un suivi des dates de péremption et de recevoir des notifications lorsque les produits approchent de leur terme. Nous mettons également à disposition la possibilité de chercher des recettes sur base du contenu de notre réfrigérateur afin de limiter le nombre d'achats à effectuer. Enfin, nous avons également une option pour proposer des aliments qui sont gustativement associés entre eux. \footnote{Si nous introduisons le mot "poulet", nous pourrions obtenir comme proposition l'épice "curry"}.

Outre ces aspects ayant attrait à la problématique que nous souhaitons traiter, notre application propose également de pouvoir scanner les barre-codes des produits afin de les insérer plus facilement à notre système ou de synchroniser le contenu du réfrigérateur au travers des différents membres d'une même "famille".

Afin d'effectuer des propositions de recettes ou d'aliments gustativement associés, nous allons étudier un vaste champ de l'informatique, celui des systèmes de recommandations. Nous présenterons un bref aperçu des différents aspects qui entrent en jeu dans cette branche récente et nous nous attarderons sur un système en particulier, celui que nous avons mis en place afin de résoudre le problème de recherche des ingrédients qui ont des affinités pour s'associer. Nous présenterons le contexte global de notre démarche et nous ne présenterons que succinctement le cas que nous avons traité afin de laisser au lecteur libre court à son imagination.